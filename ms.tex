\documentclass[letterpaper,twocolumn,showkeys]{revtex4-1}
\usepackage[utf8]{inputenc}
\usepackage{color,dcolumn,graphicx,hyperref}

\begin{document}

\title{TO FIND}

\author{Dominique Gravel}
\author{Timothée Poisot}
\author{Philippe Desjardins-Proulx}

\begin{abstract}
Background: \\
Question: \\
Approach: \\
Key results: \\
Conclusion\\

\end{abstract}

\keywords{metacommunity ecology, neutral theory, species
sorting, patch dynamics, spatial networks, dispersal}

\maketitle




%========================================================%
\section{Introduction}
%DG

- Contribution of neutral theory to ecology: the recognition of
the importance of dispersal in structuring communities. Link with limnology.


- Gave roots and motivation for the metacommunity concepts.
Present the different perspectives.


- Objective of the paper 
	- Already several reviews of neutral theory in the literature
- We will use this opportunity to explore in more details the
contribution of neutral theory to our understanding of
metacommunities

- We will pay a particular attention to recent developments at
the crossroad of network theory

- Move from a perspective where dispersal is either global or
spatially explicity (e.g. over a lattice), and spatial constant,
to a perspective focusing on the variance of dispersal.

- More specifically, the objective of the paper is to use
neutral theory to better reveal the importance of landscape
network structure on the distribution of diversity.
	- Referred as a "spatial contingency" 
	- Relevance to limnology

- Structure:
- Start with a review of the main approaches to describe spatial
networks and studies investigating them
- Then we describe three simple toy models of metacommunity
dynamics, taking this opportunity to review their assumptions
and functioning.
- R scripts for the toy models and all simulations conducted for
this paper are provided as supplementary materiel.
- We then conduct simple simulations of these models to reveal
the impact of spatial network structure on diversity
distribution.
- We conclude with a discussion on the operationally of the
framework.



%========================================================%
\section{Network representation of landscapes}
% PDP

Entités discrètes --> réseaux

 

% Table 1: main descriptors of spatial networks. Column 1: Name of the metric; Column 2: Definition
% We might want to distinguish network level versus node-level metrics.

 

%========================================================%
\section{Model description}
% DG

General introduction to the section

\subsection{Patch dynamics}
blabla

\subsection{Neutral dynamics}

\subsection{Species sorting and mass effect}
blabla

%For each model:
%	- description of the dynamics
%	- description of the algorithm
%	- main predictions
blabla

%========================================================%
\section{Results}
% TP

\subsection{Alpha diversity}

\subsection{Beta diversity} 




%========================================================%
\section{Discussion}
% TP

Contrasting the three perspectives: what do we learn?

New questions

Making the theory operational

Distribution attendue des espèces dans un paysage: utiliser la théorie neutre pour avoir un attendu (différents de la méthode classique de permutation)

- Déviation locale des prédictions

- Importance de la structure du paysage sur les propriétés émergentes

- 





\end{document}
%========================================================%