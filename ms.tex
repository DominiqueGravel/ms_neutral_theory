\documentclass[letterpaper,twocolumn,showkeys]{revtex4-1}
\usepackage[utf8]{inputenc}
\usepackage{color,dcolumn,graphicx,hyperref}

\begin{document}

\title{TO FIND}

\author{Dominique Gravel}
\author{Timothée Poisot}
\author{Philippe Desjardins-Proulx}

\begin{abstract}
Background: \\
Question: \\
Approach: \\
Key results: \\
Conclusion\\

\end{abstract}

\keywords{metacommunity ecology, neutral theory, species
sorting, patch dynamics, spatial networks, dispersal}

\maketitle




%========================================================%
\section{Introduction}
%DG

- Contribution of neutral theory to ecology: the recognition of
the importance of dispersal in structuring communities. Link with limnology.


- Gave roots and motivation for the metacommunity concepts.
Present the different perspectives.


- Objective of the paper 
	- Already several reviews of neutral theory in the literature
- We will use this opportunity to explore in more details the
contribution of neutral theory to our understanding of
metacommunities

- We will pay a particular attention to recent developments at
the crossroad of network theory

- Move from a perspective where dispersal is either global or
spatially explicity (e.g. over a lattice), and spatial constant,
to a perspective focusing on the variance of dispersal.

- More specifically, the objective of the paper is to use
neutral theory to better reveal the importance of landscape
network structure on the distribution of diversity.
	- Referred as a "spatial contingency" 
	- Relevance to limnology

- Structure:
- Start with a review of the main approaches to describe spatial
networks and studies investigating them
- Then we describe three simple toy models of metacommunity
dynamics, taking this opportunity to review their assumptions
and functioning.
- R scripts for the toy models and all simulations conducted for
this paper are provided as supplementary materiel.
- We then conduct simple simulations of these models to reveal
the impact of spatial network structure on diversity
distribution.
- We conclude with a discussion on the operationally of the
framework.



%========================================================%
\section{Network representation of landscapes}
% PDP

A network is a discrete mathematical object made of two sets: a set of 
vertices (or nodes) $V$ and a set of edges $E$ connecting the vertices
\cite{new10}. The term ``graph'' is often preferred in computer science and 
mathematics \cite{gro06}, with graph algorithms being an important and active 
area of research \cite{sed01}. A network is a combinatorial object: it is 
used to study how discrete entities are connected and how they combine 
together to create complex structures. They are used to study molecules, food 
webs, social networks, or even the relationship between variables in 
statistics \cite{wri21,new10}. We are especially interested in spatial 
networks, a special kind of network mixing the combinatorial properties of 
networks with a topological space \cite{kob94}. Thus, the vertices in a 
spatial graph are embedded in some other space, most often the two or three-
dimensional Euclidean space. This object brings a rich representation to 
spatial ecology and is particularly suited for systems of lakes and rivers, which can 
easily be represented by vertices and edges. There are two notions of 
distance in spatial networks. Euclidean distance represents the geographical 
distance between the vertices $(i, j)$, i.e.: $\sqrt{(x_i - x_j)^2 + (y_i - y_
j)^2}$. Geodesic distance is the distance in the graph space, i.e.: the 
length of the shortest path \cite{dij59}. For example, two lakes could be 
very close on a map (short Euclidean distance) but if they are not directly 
linked by a river the geodesic distance could be great.

For simulations, spatial networks can easily be generated with the random 
geometric graph algorithm \cite{sed01}. In this algorithm, all vertices are 
assigned to a position in some two-dimensional space, most often the unit 
square. Then, all pairs of vertices within some threshold Euclidean distance $r$
are connected with an edge. The resulting networks have the desirable 
property of locality: if a vertex $A$ is connected to two vertices $B$ and $C$.
then $B$ and $C$ are more likely to be connected than two random vertices. 
Random geometric networks have been extensively studied \cite{app97a,app97
b,app02a,app02b,pen03} and we provide a R function to generate them. We also 
provide the code for another structure that we call the geometric tree of 
doom. It builds a tree from the shortest path tree \cite{dij59} of a vertex in 
a random geometric network. This random geometric tree of doom looks like a
lake connected by rivers to a series of smaller lakes. 

Spatial graphs are increasingly popular in spatial ecology and conservation
biology, where patterns of connections can be used to study and influence the
flow of organisms \cite{min07,fal07,min08,gar08,urb09,dal10}. In the neutral 
theory, networks were pioneered by Economo and Keitt \cite{eco08, eco10}. 
They used networks to study how different spatial structures influenced 
diversity. They were also used to study how the spatial structure influenced 
nonsympatric speciation \cite{des12,des12b}.

% Table 1: main descriptors of spatial networks. Column 1: Name of the metric; Column 2: Definition
% We might want to distinguish network level versus node-level metrics.
Concepts
  Path                    A sequence of edges forming a sequence of vertices.
  Connection              Two vertices are connected iff there is a path between them.
  Geodesic distance       Length of the shortest path between two vertices.
Network-level metrics
  Order                   Total number of vertices.
  Size                    Total number of edges.
  Connectivity            A measure of robustness: the minimum number of elements to remove
                          to isolate the vertices.
  Components              The number of connected subsets.
Vertex-level metrics
  Degree                  The number of edges of a vertex.
  Closeness centrality    Average geodesic distance between a vertex and all other vertex.
  Eigenvector centrality  A measure of centrality based on the concept that connection to
                          highly connected vertices are more important.
  Betweenness centrality  The number of shortest paths from all vertices to all others that
                          pass through that vertex.

%========================================================%
\section{Model description}
% DG

General introduction to the section

\subsection{Patch dynamics}
blabla

\subsection{Neutral dynamics}

\subsection{Species sorting and mass effect}
blabla

%For each model:
%	- description of the dynamics
%	- description of the algorithm
%	- main predictions
blabla

%========================================================%
\section{Results}
% TP

\subsection{Alpha diversity}

\subsection{Beta diversity} 




%========================================================%
\section{Discussion}
% TP

Contrasting the three perspectives: what do we learn?

New questions

Making the theory operational

Distribution attendue des espèces dans un paysage: utiliser la théorie neutre pour avoir un attendu (différents de la méthode classique de permutation)

- Déviation locale des prédictions

- Importance de la structure du paysage sur les propriétés émergentes

- 

\bibliography{../refs}
\bibliographystyle{plain}

\end{document}
