



%--------------------------------
section{Introduction}

- Contribution of neutral theory to ecology: the recognition of the importance of dispersal in structuring communities. The provision of a  


- Gave roots and motivation for the metacommunity concepts. Present the different perspectives.


- Objective of the paper 
	- Already several reviews of neutral theory in the literature
	- We will use this opportunity to explore in more details the contribution of neutral theory to our understanding of metacommunities
	- We will pay a particular attention to recent developments at the crossroad of network theory
	- Move from a perspective where dispersal is either global or spatially explicity (e.g. over a lattice), and spatial constant, to a perspective focusing on the variance of dispersal. 
	- More specifically, the objective of the paper is to use neutral theory to better reveal the importance of landscape network structure on the distribution of diversity.
	- Referred as a "spatial contingency" 
	- Relevance to limnology

- Structure:
	- Start with a review of the main approaches to describe spatial networks and studies investigating them
	- Then we describe three simple toy models of metacommunity dynamics, taking this opportunity to review their assumptions and functioning.
	- R scripts for the toy models and all simulations conducted for this paper are provided as supplementary materiel.
	- We then conduct simple simulations of these models to reveal the impact of spatial network structure on diversity distribution. 
	- We conclude with a discussion on the operationally of the framework.



%--------------------------------
section{Network representation of landscapes}


% Table 1: main descriptors of spatial networks. Column 1: Name of the metric; Column 2: Definition
% We might want to distinguish network level versus node-level metrics.

 

%--------------------------------
section{Model description}


For each model:
	- description of the dynamics
	- description of the algorithm
	- main predictions


%--------------------------------
section{Results}

- Alpha diversity
- Beta diversity 




%--------------------------------
section{Conclusion}

