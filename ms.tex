\documentclass[12pt]{article}
\usepackage[utf8]{inputenc}
\usepackage{color,lineno,setspace,graphicx,multirow,kpfonts}
\usepackage[top=2.4cm,left=2.4cm,top=2.4cm,bottom=2.4cm,includefoot]{geometry}
\usepackage[style=bes]{biblatex}
\bibliography{./library}

\begin{document}

\linenumbers 
\modulolinenumbers[1]

\textbf{Title:}   Using neutral theory to reveal the contribution of dispersal to community assembly\\

\textbf{Authors:}  Dominique Gravel$^{1,2,*}$, Timoth\'ee Poisot$^{1,2}$, Philippe Desjardins-Proulx$^{1,2}$\\

1: Canada Research Chair on Terrestrial Ecosystems. D\'epartement de biologie, chimie et g\'eographique, Universit\'e du Qu\'ebec \`a Rimouski, 300 All\'ee des Ursulines, Qu\'ebec, Canada. G5L 3A1.\\

2: Qu\'ebec Centre for Biodiversity Sciences, Stewart Biological Sciences Building, 1205 Dr.~Penfield Avenue, Montr\'eal (QC), H3A 1B1, Canada\\

\textbf{Words in the abstract:}      \\
\textbf{Words in the main text:}    \\
\textbf{Words in the legends:}    \\
\textbf{References:}             \\
\textbf{Table:}                    \\

\newpage
\doublespacing

%========================================================%

\section*{Abstract}
Background: \\
Question: \\
Approach: \\
Key results: \\
Conclusion\\

\textbf{Keywords:} metacommunity ecology, neutral theory, species
sorting, patch dynamics, spatial networks, dispersal\\

\newpage

%========================================================%
\section{Introduction}

Community assembly is concerned by patterns and processes occurring at various
spatial scales \parencite{Levin1992}. Until the development of metacommunity
ecology, studies on community assembly were often restricted to local
populations, with a strong focus on pairwise interactions (e.g.
\textcite{MacArthur1972, May1973,Pimm1982,DeAngelis1992}). The emphasis on local
communities has been vigorously criticized by \textcite{Ricklefs2008}, who has
long recognized that local dynamics and community structure are strongly
contingent on processes occurring at much larger spatial scales
\parencite{Ricklefs1987}. This perspective is particularly relevant to
limnology, where exchanges of organisms and nutrients affect community and
ecosystem properties from the local (e.g. vertical mixing
\parencite{Ryabov2011}) to the regional (e.g. connection of lakes
\parencite{Leibold2004b} scales. It emphasizes the importance of dispersal
relative to pairwise interactions in the organization of ecological communities.
The metacommunity concept has been proposed by \textcite{Leibold2004a} as a
novel approach to link different spatial scales in ecology. It builds on
feedbacks between local scale processes, such as competitive interactions and
local adaptation, and regional scale processes such as dispersal, gene flow and
speciation. Ecologists are now required to move toward a predictive ecology,
integrating elements of theoretical ecology \parencite{Thuiller2013}, and the
metacommunity perspective appears naturally as the appropriate conceptual
framework to develop the new modeling techniques required to fill this
challenge. The development of neutral theory has been quite provocative in that
respect, as one could see it as a step back in time. Neutral theory makes the
provocative assumption that species are ecologically equivalent and thereby any
variation in the environment has no impact on demography
\parencite{Bell2000,Hubbell2001}. Only demographic stochasticity and dispersal
drive the structure of neutral ecological communities. It therefore appears
that, on first sight, neutral theory is useless.  We will develop in this paper
the argument that neutral theory could be a useful tool to understand the impact
of dispersal on community organization in landscapes of various complexities. 

Neutral theory sparked an historical debate in community ecology that is still
lasting after more than a decade \parencite{Chave2004, Etiennee2011,
Rosindell2012,Clark2012}. It was stimulated by the impressive ability of neutral
models to fit several well studied empirical observations such as species
abundance distributions and distance-decay relationships. A remarkable strenght
of neutral theory is to provide a \emph{"formal general theory of abundance
and diversity that will account, in a simple and economical fashion, for the
many patterns that ecologists have documented"} \parencite{Bell2001}. Even if
new studies rejecting neutral theory are consistently published (e.g.
\textcite{Ricklefs2012}), there is now almost a consensus that neutral theory is
a well-developed null hypothesis for niche theory and could even be used as an
adequate approximation of ecological dynamics in some situations.
\textcite{Bell2001} nicely envisionned two perspectives to neutral theory that
are still standing today. Under the weak perspective, neutral theory provides a
set of realistic predictions of community organization despite false
assumptions. Even if being fundamentally wrong, neutral theory is still useful
when used as a null hypothesis \parencite{Gotelli2006}. It is considered as an
improvement over traditional null hypotheses based on randomization
\parencite{Gotelli2000} because it readily integrates dispersal. The strong
version on the other hand posits that neutral theory is a satisfying
approximation to community dynamics and an appropriate theory to explain the
distribution of biodiversity. It implies that the right mechanisms have been
identified and that the consistently observed differences among species do not
impact community organization. 

Neutral theory has also been proposed as a useful tool to understand and predict
some aspects of community dynamics. It links to an old philosophical debate
between realism and instrumentalism \parencite{Wennekes2011}. Because every
ecological model is a simplification of reality, any scientist has to
subjectively decide the level of details he puts in, leaving out some elements
judged unimportant. The realism perspective requires that all assumptions of
theory to be true, while the utility of the theory is more important to
instrumentalism. The utilitarian value of a theory could either be for
understanding or prediction (another old philosophical debate, see
\textcite{Schmueli2010}). Obviously neutral theory could only be instrumental.
The question then is if such a 'general, large-scale, but vague' theory
\parencite{Wennekes2011} is a satisfying approximation. 

The instrumentalism view of neutral theory raises the question of why should it
be a satisfying approximation despite knowing the pieces are wrong? We see two
potential answers to this question. A first answer might be that stochasticity
of various origins can blur the deterministic differences among species and
promote the ecological drift \parencite{Gravel2011}. Much has been said the
existence of demographic stochasticity, some ecologists even arguing that
neutral models impede progress in community ecology by hidding niche differences
\parencite{Clark2012}, and we therefore will keep this discussion for other
papers. The second answer is that dispersal and historical contingencies might
have a much more profound impact on species distribution \parencite{Bahn2007}
and ecological dynamics. The debate over the equivalence assumption and
demographic stochasticity has perhaps overlook the recognition of how much
dispersal influence community assembly.

We will adopt the perspective that neutral theory is a useful tool to both
understand and predict the impact of dispersal on community organization. Even
for a theoretical analysis, we need a benchmark without niche differences to
reveal the role of dispersal in structuring communities and understand the
interaction with niche differentiation. We will explore recent applications of
neutral theory, at the crossroad of network theory, to better represent the
impact of landscape structure on biodiversitity distribution. This analysis will
prove particularly relevant to limnology, where most riverine and lacustre
habitats are characterized by a their discrete nature and spatially complex
arrangements. We will also reveal the relative contribution of ecological
interaction and niche differentiation by contrasting predictions of a neutral
model to other metacommunity perspectives. 

Our main objective in this paper is to use neutral theory to stress the
importance of landscape network structure on the distribution of diversity. We
refer to the landscape organization as a "spatial contingency"
\parencite{Peres-Neto2013} that could potentially affect the coexistence
mechanisms at play. We will therefore move from a perspective where dispersal is
either global or spatially explicity (e.g. over a lattice), and spatial
constant, to a perspective focusing on the variance of dispersal. A second
generation of neutral models (e.g. \textcite{Economo2008,
Economo2011,Desjardins2012a,Desjardins2012b}, and even experiments
\parencite{Altermat2012}, recently introduced more realistic lanscapes and found
surprising contributions of spatial contingencies. We will start with a short
review of the main approaches to describe spatial networks and the studies
investigating them. Then we will describe three simple toy models of
metacommunity dynamics, taking this opportunity to review their assumptions and
main predictions. We provide as Supplementary Material the R scripts for the toy
models and all simulations conducted for this paper. We then conduct simple
simulations of these models to reveal the impact of spatial network structure on
diversity distribution. We conclude with a discussion on the operationally of
the framework.

%========================================================%
\section{Network representation of landscapes}
% PDP

Entités discrètes --> réseaux

 

% Table 1: main descriptors of spatial networks. Column 1: Name of the metric; Column 2: Definition
% We might want to distinguish network level versus node-level metrics.

 

%========================================================%
\section{Model description}
% DG

General introduction to the section

\subsection{Patch dynamics}
blabla

\subsection{Neutral dynamics}

\subsection{Species sorting and mass effect}
blabla

%For each model:
%	- description of the dynamics
%	- description of the algorithm
%	- main predictions
blabla

%========================================================%
\section{Results}
% TP

\subsection{Alpha diversity}

\subsection{Beta diversity} 




%========================================================%
\section{Discussion}
% TP

Contrasting the three perspectives: what do we learn?

New questions

Making the theory operational
	- how to parameterize neutral models
	- making predictions for specific landscapes
	- 


Distribution attendue des espèces dans un paysage: utiliser la théorie neutre pour avoir un attendu (différents de la méthode classique de permutation)

- Utiliser les déviations locales pour comprendre le rôle de la sélection 

- Approche par réseau: vers une approche qui intègre la variance des mécanismes de coexistence

- Importance de la structure du paysage sur les propriétés émergentes
	- importance relative de species sorting et drift va dépendre de l'organisation du paysage
	- prédiction
	- rôle de la distribution de degrés
- 

%========================================================%
\section{Acknowledgements}
DG received financial support from NSERC and Canada Research Chair program. TP is funded by a MELS-FQRNT post-doctoral fellowship and PDP by a NSERC fellowship.
\newpage

%========================================================%
\printbibliography
\newpage



\end{document}
%========================================================%
