\documentclass[12pt]{article}
\usepackage[utf8]{inputenc}
\usepackage{color,lineno,setspace,multirow,kpfonts}
\usepackage{graphicx}
\usepackage[top=2.4cm,left=2.4cm,top=2.4cm,bottom=2.4cm,includefoot]{geometry}
\usepackage[style=bes]{biblatex}
\bibliography{library}

\begin{document}

\linenumbers 
\modulolinenumbers[1]

\textbf{Title:}  Using neutral theory to reveal the contribution of dispersal to community assembly in complex landscapes\\

\textbf{Authors:}  Dominique Gravel$^{1,2,*}$, Timoth\'ee Poisot$^{1,2}$, Philippe Desjardins-Proulx$^{1,2}$\\

1: Canada Research Chair on Terrestrial Ecosystems. D\'epartement de biologie, chimie et g\'eographique, Universit\'e du Qu\'ebec \`a Rimouski, 300 All\'ee des Ursulines, Qu\'ebec, Canada. G5L 3A1.\\

2: Qu\'ebec Centre for Biodiversity Sciences\\

\textbf{Keywords:} metacommunity; neutral theory; species sorting; metapopulation; spatial network; centrality \\

\textbf{Words in the abstract:}      

\textbf{Words in the main text:}    

\textbf{Words in the legends:}    

\textbf{Figures:} 4 

\textbf{Tables:} 2                    

\textbf{References:}             

\newpage
\doublespacing

%========================================================%
\section*{Abstract}
The metacommunity perspective appears naturally as the appropriate conceptual
framework to make ecology more predictive, integrating elements of theoretical
ecology. The recent development of neutral theory appears as a step back in that
direction because of the assumption of ecological equivalence and the absence of
any effect of the environment on community organization. A remarkable strength
of neutral theory is nonetheless to provide a general theory of diversity that
accounts for a wide range of empirical observations. In this paper, we argue
that neutral theory can be a useful tool to understand the impact of dispersal
on community organization in landscapes of various complexities. Our main
objective is to use neutral theory to stress the importance of complex landscape
network structure on the distribution of diversity. We refer to the landscape
organization as a "spatial contingency" that could potentially affect the
coexistence mechanisms at play. We briefly review the main approaches to
describe spatial networks and describe three simple toy models of metacommunity
dynamics. We take this opportunity to review their assumptions and main
predictions. We then conduct simple simulations of these models to reveal with
simple examples the impact of spatial network structure on diversity
distribution. The simulation results show that competitive interactions buffer
the potential impact of landscape structure. The strongest centrality-species
richness relationship was observed for the patch dynamics, a model without any
interactions. On the other hand, strong and unequal competitive interactions
minimized the effect of centrality. We conclude that the neutral model is thus a
useful tool to understand the joint effects of dispersal and community
interactions. Our analysis shows that ecologists must now integrate more
realistic landscapes when analyzing community assembly from a metacommunity
perspective.
\newpage

%========================================================%
\section*{Introduction}

Ecology needs to move toward a more predictive approach, integrating elements of
theoretical ecology \parencite{Thuiller2013}. The metacommunity perspective
\parencite{Leibold2004a} appears naturally as the appropriate conceptual
framework to fill this challenge. The metacommunity concept builds on feedbacks
between local scale processes, such as competitive interactions and local
adaptation, and regional scale processes such as dispersal, gene flow and
speciation. It is particularly relevant to limnology, where exchanges of
organisms and nutrients affect community and ecosystem properties from the local
(e.g. vertical mixing \parencite{Ryabov2011}) to the regional (e.g. connection
of lakes \parencite{Leibold2004b, Gravel2010a} scales. It emphasizes the
importance of dispersal relative to pairwise interactions in the organization of
ecological communities.

At first sight, the development of neutral theory appears as a step back.
Neutral theory of biodiversity makes the provocative assumption that species are
ecologically equivalent \parencite{Bell2000,Hubbell2001}. Neutral ecological
communities are driven only by demographic stochasticity and dispersa and thus,
variation in the environment has no impact on demography. Neutral theory sparked
an historical debate still lasting after more than a decade
\parencite{Chave2004, Etienne2011, Rosindell2012,Clark2012}. It was stimulated
by the surprising ability of neutral models to fit some well studied empirical
observations such as species abundance distributions and distance-decay
relationships.

A remarkable strength of the theory is to provide a \emph{"formal general theory
of abundance and diversity that will account, in a simple and economical
fashion, for the many patterns that ecologists have documented"}
\parencite{Bell2001}. Even if new studies rejecting neutral theory are
consistently published (e.g. \textcite{Ricklefs2012}), a consensus is forming
that neutral theory is a well-developed null hypothesis for niche theory and
could even be used as an adequate approximation of ecological dynamics in some
situations. \textcite{Bell2001} nicely envisioned two perspectives to neutral
theory that are still standing today. Under the weak perspective, neutral theory
provides a set of realistic predictions of community organization despite false
assumptions. Even if being fundamentally wrong, neutral theory would still
useful when used as a null hypothesis \parencite{Gotelli2006}. It is considered
as an improvement over traditional null hypotheses based on randomization
\parencite{Gotelli2000} because it readily integrates dispersal. On the other
had, the strong version posits that neutral theory is a satisfying approximation
to community dynamics and an appropriate theory to explain the distribution of
biodiversity. It implies that the right mechanisms have been identified and that
the consistently observed differences among species do not impact community
organization.

Neutral theory has also been proposed as a useful tool to understand and predict
some aspects of community dynamics. It links to an old philosophical debate
between realism and instrumentalism \parencite{Wennekes2012}. Because every
ecological model is a simplification of reality, scientists have to subjectively
decide the level of details they put in, leaving out some elements they consider
unimportant. The realism perspective requires that all assumptions of the theory
must be true, while the utility of the theory is more important to
instrumentalism. The utilitarian value of a theory could either be for
understanding or for prediction (another old philosophical debate, see
\textcite{Shmueli2010}). Obviously neutral theory could only be instrumental.
The question then is if such a 'general, large-scale, but vague' theory
\parencite{Wennekes2012} is a satisfying approximation.

The instrumentalist view of neutral theory raises the question of why it should
be a satisfying approximation despite knowing the pieces are wrong? Perhaps
stochasticity of various origins blur the deterministic differences among
species and promote ecological drift \parencite{Gravel2011a}. Much has been said
the existence of demographic stochasticity \parencite{Clark2012}, and we
therefore will let this discussion for other papers. A second answer is that
dispersal and historical contingencies might have a much more profound impact on
species distribution \parencite{Bahn2007, Boulangeat2012} and ecological
dynamics. The debate over the equivalence assumption and demographic
stochasticity has perhaps overlook the recognition of how much dispersal
influence community assembly.

In this paper, we argue that neutral theory can be a useful tool to understand
the impact of dispersal on community organization in landscapes of various
complexities. Even for purely theoretical analyses, we need a benchmark without
niche differences to reveal the role of dispersal in structuring communities and
understand how it interacts with niche differentiation. We will explore recent
applications of neutral theory, at the crossroad of network theory, to better
represent the impact of landscape structure on biodiversity distribution. This
analysis will prove particularly relevant to limnology, where most riverine and
lacustre habitats are characterized by their discrete nature and spatially
complex arrangements \parencite{Peterson2013}. We will also explore the relative
contribution of ecological interaction and niche differentiation by contrasting
predictions of a neutral model to other metacommunity perspectives.

Our main objective is to use neutral theory to stress the importance of
landscape network structure on the distribution of diversity. We refer to the
landscape organization as a \emph{spatial contingency} \parencite{Peres-Neto2012}
that could potentially affect the coexistence mechanisms at play. We will
therefore move from a perspective where dispersal is either global or spatially
explicit (e.g. over a lattice), and spatial constant, to a perspective focusing
on the variance of dispersal. A second generation of neutral models (e.g.
\textcite{Economo2008, Economo2011,Desjardins2012a,Desjardins2012b}), and even
experiments \parencite{Carrara2012}, recently introduced more realistic
landscapes and found surprising contributions of spatial contingencies. We start
with a short review of the main approaches to describe spatial networks. Then we
describe three simple toy models of metacommunity dynamics, using this
opportunity to review their assumptions and main predictions. We provide as
Supplementary Material the R scripts for the toy models and all simulations
conducted for this paper. We then conduct simple simulations of these models to
reveal with simple examples the impact of spatial network structure on diversity
distribution. We conclude with a discussion on the operationally of the
framework.

%========================================================%
\section*{Network representation of landscapes}

A network is a discrete mathematical object made of two sets: a set of nodes
(or vertices) and a set of edges connecting the nodes \parencite{new10}. The
term ``graph'' is often preferred in computer science and mathematics
\parencite{gro06}, with graph algorithms being an important and active area of
research \parencite{sed01}. A network is a combinatorial object: it is used to study
how discrete entities are connected and how they combine together to create
complex structures. They are used to study molecules, food webs, social
networks, or even the relationship between variables in statistics
\parencite{wri21,new10}. We are especially interested in spatial networks, a special
kind of network mixing the combinatorial properties of networks with a
topological space \parencite{kob94}. Thus, the vertices in a spatial graph are
embedded in some other space, most often the two or three- dimensional Euclidean
space. This object brings a rich representation to spatial ecology and is
particularly suited for systems of lakes and rivers, which can easily be
represented by vertices and edges. There are two notions of distance in spatial
networks. Euclidean distance represents the geographical distance between the
nodes $(i, j)$, i.e.: $\sqrt{(x_i - x_j)^2 + (y_i - y_ j)^2}$. Geodesic
distance is the distance in the graph space, i.e.: the length of the shortest
path \parencite{dij59}. For example, two lakes could be very close on a map (short
Euclidean distance) but the geodesic distance could be great if they are not
directly linked by a river.

The popularity of network theory stems from its ability to model complex
structures while allowing us to extract useful metrics (Table 1). At a very high
level, a network can be described by its number of nodes (the order) and 
edges (the size). Looking more closely, the relationship between
nodes is influenced by paths, which are ordered series of nodes. Centrality
is a \emph{central} concept in network theory, where it can be seen
to as a measure of ``importance''. The simplest measure of the centrality of a
node is its degree, which is the number of nodes directly connected to it. Of
course, this is a very rough description of centrality. For example, two lakes
can have the same degree, with one being connected to a small isolated cluster,
while the other one is part of one of the biggest network of lake. In this case,
measures of centrality like eigen-centrality will weight the importance of the
connection, so a node connected to well-connected nodes will have higher
centrality than a node connected to isolated nodes.

For simulations, spatial networks can easily be generated with the random
geometric graph algorithm \parencite{sed01}. In this algorithm, all nodes are
assigned to a position in some two-dimensional space, most often the unit
square. Then, all pairs of nodes within some threshold Euclidean distance $r$
are connected with an edge. The resulting networks have the desirable property
of locality: if a node $A$ is connected to two vertices $B$ and $C$. then $B$
and $C$ are more likely to be connected than two random vertices. Random
geometric networks have been extensively studied \parencite{app97a,app97b,app02a,app02b,pen03} 
and we provide a R function to generate them. The
position of nodes is typically random, but we could also imagine alterations
where they are either more aggregated or segregated than expected by chance
alone.

We also provide the code for a second structure that we call a random geometric
tree. The algorithm first builds a random geometric graph, then select a node
from which to start the tree. It then calculaes the the shortest path tree
\parencite{dij59} from this node to all other ones. This random geometric tree
does not exactly represent dendritic landscapes but is a convenient model to
simulate a lake connected by rivers to a series of smaller lakes.

Spatial graphs are increasingly popular in spatial ecology and conservation
biology, where patterns of connections can be used to study and influence the
flow of organisms \parencite{min07,fal07,min08,gar08,urb09,dal10}. In the neutral
theory, networks were pioneered by Economo and Keitt \parencite{Economo2008, eco10}. They
used networks to study how different spatial structures influenced diversity.
They were also used to study how the spatial structure influenced nonsympatric
speciation \parencite{Desjardins2012a,Desjardins2012b}.

%========================================================%
\section*{Model description}

In this section we describe three toy models representing different
perspectives of metacommunity ecology: patch dynamics, neutral dynamics and
species sorting. While the neutral model is interesting in itself, it is by its
comparison with a model without any interactions (patch dynamics) and with niche
differentiation (species sorting) that we will be able to fully understand the
interaction between these processes and landscape structure. Despite neutral,
competitive interactions in neutral models are very strong because of the
zero-sum assumption (the community is always at carrying capacity). We will
first review the fundamental assumptions of each model with their description
(Table 2 summarizes the parameters and variables that are used), and then
briefly discuss their main predictions. Simulation results are presented in the
next section, with the corresponding R code provided in the Supplementary
Material.

%------------------------
\subsection*{Patch dynamics}

The simplest metacommunity model is a $S$ species extension of traditional
metapopulation models \parencite(Hanski1999). The standard Levins metapopulation
model \parencite(Levins1969) describes the stochastic colonizations and
extinctions of a single species over a homogenous landscape. The basic unit is
the population. The model tracts the dynamics of occupancy (the fraction of the
landscape that is occupied) with an ordinary differential equation and therefore
assumes an infinite landscape. The simulation model we run is more realistic as
it simulates a finite number $N$ of discrete patches (or nodes in network
terminology). The rules described in the previous section were used to generate
connectivity matrices along four scenarios (Fig. 1): global dispersal (connected
graph), a lattice, a random geometric graph and a random tree graph. A patch $x$
shares $d_x$ links with neigbhouring patches (its degree). At each time step
(the simulation model is discrete in time), the probability that a colonist
coming from an occupied patch $y$ arrives at patch $x$ is $cd_y^{-1}$, where $c$
is the probability a colonization event takes place if all connected patches are
occupied. The expected probability that a colonist arrives to patch $x$ from
patch $y$ is then $C_{ixy}=cp_{iy}d_y^{-1}$, where $p_{i}y$ is the probability
that patch $y$ is occupied by species $i$. The probability that an extinction
occurs in a given patch is $e$. The Levins model is for a single species, but a
basic metacommunity patch dynamics model could be run by aggregating $S$
independent metapopulation models \parencite(Hanski1997). There are no
interactions in this simple model, which means there is no limit to local
species richness and no carrying capacity. Competitive, mutualistic and
predator-prey interactions have been added to this framework (e.g.
\textcite{Tilman1994a,Holt1996,Klausmeier1998,Gravel2011b}) but we will keep this
model minimal for the sake of comparison with the neutral model.

Predictions of the patch dynamics metacommunity model are quite straightforward.
First, a fundamental result of metapopulation ecology is that persistence will
occur if colonization probability is larger than extinction probability ($c>e$).
Given that all species are the same, then we should expect the regional
diversity ($\gamma$) to be $S$ if this condition is satisfied and $0$ if not.
The situation is however more complex in spatially explicit landscapes with
complex connectivity matrices \parencite{Hanski1998}. Spatially explicit
dispersal usually reduces the occupancy and thereby the likelihood of
persistence. The second prediction is that, given spatial variation in
connectivity, there will be spatial variation in occurrence probability. Given
the above formulation of a colonization event to occur, the probability that an
empty location is colonized is $I_ix=1-\prod d_x(1-C_{ixy})$. This equation
basically tells us that the colonization probability will increase
asymptotically with the degree of a patch (because of the product). It is easy
to show from metapopulation theory that the occurrence probability in a patch is
then $p_ix=I_x(I_ix+e)^{-1}$. The feedback between local and regional dynamics
arises because all $p_ix$ from the landscape are dependent from each other.
Simulations are usually conducted to solve the model for a large landscape, but
numerical solutions are theoretically possible. The aggregation across the $S$
species of the regional species pool is obtained by taking the summation of
occurrence probabilities over all species, $s_x = \sum{p_i}$. Because in this
model all species are equal, we expect the local species richness to be a linear
function of the patch degree (number of edges). Multi-species analysis of
metapopulation models also reveals interesting predictions on other aspects of
community organization at various spatial scales such as the species-area
relationship \parencite{Hanski1997}, and proved to be useful in conservation
ecology with predictions of extinctions following habitat destruction
\parencite{Tilman1994b,Rybicki2013}.

%------------------------
\subsection*{Neutral dynamics}

Neutral theory introduces strong competitive interactions by assuming there is a
finite number of individuals that could occupy a patch. There are different ways
to simulate this \emph{zero-sum rule} \parencite{Bell2000,Hubbell2001}, but they
all result in the same constraint that the increase in abundance of a species
could only occur after an equivalent decrease by another species. One important
change in the formulation of most neutral models relative the patch dynamics
model presented above is therefore that it is individual-based, not population
based. We therefore considered in our toy model of neutral dynamics that each
local patch holds $J_x$ individuals. The model tracts the local abundance of all
species $N_{ix}$ in each local patch. At each time step an individidual dies
with probability $k$. Recruitment only occurs in vacant sites, similarly to a
tree by tree replacement process in a closed canopy forest.

The formulation of the recruitment probability is the central piece of all
neutral models, making possible the coupling with the metacommunity and
neighbouring patches. We adopt a simple formulation based on
\parencite{Gravel2006}. The approach is conceptually similar to placing a trap
in a canopy gap and picking a seed at random among the ones falling in to
determine the identity of the recruited species. The composition of the seed
pool in that trap will be a mixture of local dispersal and immigrants from the
metacommunity. For simplicity, we consider three spatial scales of dispersal but
it would be easy to generalize the approach to a continuous seed dispersal
kernel \parencite{Gravel2006}. The parameter $m$ is the probability that the
recruit is a migrant from neighbouring patches, $M$ is the probability it comes
from a larger (and fixed) metacommunity, and consequently, by substraction,
$1-m-M$ is the probability it comes from local dispersal. The fraction
$N_{ix}J_x^{-1}$ is the local relative abundance and $P_ix$ is the relative
abundance of species $i$ in the seed pool coming from neigbouring patches $x$.
The relative abundance in the neighborhood is weighted by the degree of the
connected nodes because some nodes will spread their seeds across a higher
number of nodes and thus contribute less to the seed pool. We thus consider
$P_{ix} =\frac{\sum P_{iy}d_y^{-1}}{\sum d_y^{-1}}$. We assume for simplicity
(and without loss of generality, \parencite{Bell2000}) that the relative
abundance in the metacommunity is uniform, i.-e. equal to $S^{-1}$. This
immigration prevents the collapse of the metacommunity to a single species,
since otherwise all species except one will face extinction by ecological drift
(speciation prevents this phenomenon to occur in \textcite{Hubbell2001}). The
local recruitment probability is consequently $Pr_{ix} = MS^{-1} + mP_{ix} +
(1-m-M)N_{ix}J_x^{-1}$.

The model is neutral because it assumes that the probabilities of local
recruitment, immigration and mortality events are all equal across species.
Demographic stochasticity is the source of variations in abundance, but larger
disturbances could be simulated as well, as long as they hit all species with
the same probability, independently of their abundance. The fundamental feature
of neutral dynamics is therefore the ecological drift, defined as population
changes emerging from neutraly stable population dynamics. It can be measured as
the variance between replicated time series of community dynamics
\parencite{Gravel2011a}. \textcite{Hubbell2001} provides a very comprehensive
analysis of the model, with specific attention to the effect of the different
parameters on drift (and consequently variance in abundance) and time to
extinction. Despite its simplicity, the neutral model is surprisingly rich in
the predictions it makes. \textcite{Bell2001} and \textcite{Hubbell2001}
analyzed the performance of neutral models to predict species abundance
distributions, the range-abundance relationship, spatial variation in abundance,
the species-area relationship, community turnover (beta-diversity) and
co-occurrence. Recent trophic neutral models were also found to predict
realistic ecological network structures \parencite{Canard2012}. Other than the
ecological equivalence assumption, one of the most criticized aspect of neutral
models is the realism of the speciation process and the required speciation
rates to sustain species richness \parencite{Ricklefs2003,Etienne2007}. Recent
neutral models with more credible speciation models
\parencite{Rosindell2010,Desjardins2012a} revealed the difficulty to maintain
diversity in neutral models over macro-evolutionary time scales. These models
nonetheless proposed interesting predictions on endemic species richness and
island biogeography \parencite{Rosindell2011,Desjardins2012b}.

%------------------------
\subsection*{Species-sorting and mass effect}

The species-sorting and the mass effect perspectives build on the notion of
species-specific responses to a spatially varying environment
\parencite{Leibold2004a}. There are various ways to simulate such dynamics and
we picked the lottery model, in line with tradition \parencite{Mouquet2002} and
for its proximity to the neutral model described above. Competition for space
occurs during recruitment after the death of an adult. The recruitment is a
lottery among potential candidates as in the neutral model. The recruitment probability 
is however biased by species specific responses to local environmental conditions. 

The lottery dynamics described above for the neutral model assume there is a
very large number of offsprings that are candidate for recruitment but only one
will survive and develop to the adult stage. The effect of a differentiation to
local environmental conditions could be implemented at this stage with a biased
survival probability. The $J_x$ individuals all experience a unique
environmental condition $E_{nx}$ called a microsite $n$. We considered a patch
average $\overline{E_x}$, with a within-patch variance $\sigma_x$. The regional
average is $\overline{E_R}$ and the regional variance $\sigma_R$ (for simplicity
we considered normal distributions of environmental conditions, but different
distributions will lead to different regional similarity constraints
\parencite{Mouquet2003,Tilman2004,Gravel2006}). We consider that a fraction
$\lambda_{inx}$ of offsprings reaching the microsite where recruitment occurs
will survive. The recruitment probability is therefore biased in favour of the
species with highest survival because only some species will be able to cope
with the local environmental conditions. We define the relative
abundance in the seed rain as $Z_{ix} = MS^{-1} + mP_{ix} +
(1-m-M)N_{ix}J_x^{-1}$. The calculation of the relative abundance in the seed
rain is the same as the neutral model but the recruitment probability differs
because only a fraction of offspring survive. It is formulated as $Pr_{ix} =
\frac{\lambda_{inx}Z_{ix}}{\sum \lambda_{jnx}Z_{jx}}$. The function describing
the relationship between a microsite condition and survival could take various
forms; we used the traditional gaussian curve describing the niche,
$\lambda_{inx} = \exp{-\frac{(E_{nx}-u_i)^2}{2\Pi b_i^2}}$, where $u_i$ is the
niche optimum and $b_i$ is niche breadth. Note that the model will converge to a
neutral model when the niche breadth tends to infinity (which is in fact how we
simulated neutral dynamics in the Supplementary Material to minimize the
complexity of the code).

Analyses of similar models with a combination of dispersal and species-sorting
shown that predictions are extremely variables and depend on the frequency
distributions of environmental conditions, niche optimums and breadth. For
instance, a well-studied prediction of neutral models is the species abundance
distribution. It was shown that niche models can predict similar distributions
given appropriate parameters \parencite{Tilman2004,Gravel2006}. The main
prediction is nonetheless that stable and predictable (meaning which species
will coexist) if species are sufficiently dissimilar, which differs from neutral
models. Local species richness will first depend on the joint effects of local
heterogeneity and niche breadth because coexistence requires a sufficient
dissimilaritya mong species \parencite{Schwilk2005}. Local species richness
could be increased by a mass effect when dispersal is consistently supplying
individuals coming from more favorable locations (refuges). The limiting
similarity required to maintain regional coexistence depends on the amount of
dispersal because exchanges among communities homogenizes environmental
conditions. This is one of the main result from the species sorting theory and a
clever example of local-regional feedbacks: increasing dispersal promotes local
coexistence, but on the other hand it diminishes regional coexistence. Only the
best average competitors will remain at very high dispersal. We therefore expect
a hump-shaped relationship between dispersal and alpha ($\alpha$) diversity,
with a peak at intermediate dispersal. On the other hand, we expect a monotonic
decrease of beta ($\beta$) and $\gamma$ diversity with dispersal
\parencite{Mouquet2003}. This prediction has been validated in some experiments
\parencite{Venail2008, Logue2011}.

\section*{Results}

In this section we provide simple simulation results to illustrate the impact of
saptial contingencies on species distribution and coexistence. We consider four
different landscapes, illustrated at Fig. 1. with the outcome of simulations
using the neutral model. All of these networks have the same number of nodes
(e.g. spatial sampling sites), but both different number of edges (e.g.
dispersal routes between sampling sites) and patterns of connectivity between
nodes. We ask how these differences in connectivity will shape the emerging
properties of the community under the scenarios represented by each
metacommunity model. Our analysis is not exhaustive, it is provided simply to
illustrate the interaction between metacommunity perspectives and landscape
structures on $\alpha$, $\beta$ and $\gamma$ diversity.

In Fig. 2, we present the species richness of each node of the network ($\alpha$
diversity), as a function of the centrality of the node, under different
assumptions of metacommunity dynamics and network structure. We scaled the
species richness by the maximal $\alpha$ diversity to facilitate comparison
between models. The model parameterization is responsible for differences in
both $\alpha$ and $\gamma$ diversity, meaning that only the shape of the
relationship between centrality and richness ought to be looked at. It appears
that both in the geographical and tree graph, the path dynamics model has a
much more considerable variation in $\alpha$ diversity.  However, in all
cases the $\alpha$ diversity increases with the node degree centrality, meaning
that nodes with more connections also host a more diverse community.
Eigen-centrality gave a far less clear-cut result, which can probably be
attributed to the fact that our networks are relatively small in size.
Eigen-centrality reports how well your neighbors are connected, and in graphs
with a short diameter (i.e. the two farthest points are not extremely far
apart), this measure might hold less information.

Finally, Figs. 3 and 4 present, respectively, the between patch $\beta$
diversity as a function of the shape of the network, under the three dynamic
models. We used Bray-Curtis measure of dissimilarity between patches. In Fig. 3,
the distance is expressed as the Euclidean (geographic) distance between two
patches. Although this neglects how dispersal connects the different patches,
there is already a clear signal of geographic distance on $\beta$ diversity,
indicating the importance of dispersal under the three scenarios. In both the
neutral and patch dynamics model, local communities become increasingly
dissimilar when the distance between them increases. In other words, two
communities which are close to each other will share a large proportion of their
species pool, whereas two communities which are afar will share a small
proportion. The relationship between distance and dissimilarity is similar for
species-sorting. Nonetheless, it forms an enveloppe of points (with most points
lying in the upper-left part of the graph). While two distant communities will
be dissimilar, there is no telling how dissimilar two close communities will be.
Note this relationship for species-sorting varies significantly with the spatial
distribution of microsites (not shown). At one extreme, if all patches hold the
same average conditions, then we should expect no relationship between
dissimilarity and distance. On the other hand, if the average conditions are
highly variable among localities (as in here), then we should expect two
communities close to be potentially dissimilar (if conditions are different) or
similar (if they are the same). The variance should thus be larger. A
distance-dissimilarity relationship arises in the situation where dispersal
promotes a mass effect (as in here). Such results emphasize the interaction
between spatial contingencies (here connectivity and distribution of
environmental conditions) and dispersal.

To a vast extent, these relationships are preserved when looking at the
geodesic distances (Fiig. 4), i.e. along how many edges should one travel to
connect two patchs. Interestingly enough, the distance-dissimilarity
relationship for the neutral model is markedly hump-shaped, with sites being at
a medium distance having the maximal dissimilarity.

%========================================================%
\section*{Discussion}

Our objective in this paper was to review the main assumptions of three
metacommunity models and illustrate how the implementation of more realistic
landscapes could reveal the importance of dispersal on community structure. We
argued in the introduction that neutral theory is useful both to understand and
predict the impact of dispersal on community organization. The review of the
different models shows that the fundamental difference between a neutral model
and the patch dynamics model is the effect of competitive interactions on
distribution, while the difference between the neutral and the species sorting
models is the effect of unequal competitive interactions. The neutral model is
thus a useful tool to understand the joint effects of dispersal and community
interactions. Our comparison of the distribution of $\alpha$ diversity was
particularly meaningful in that respect. The simulation results show that
competitive interactions buffer the potential impact of landscape structure. The
strongest centrality-species richness relationship was observed for the patch
dynamics, a model without any interactions. On the other hand, strong and
unequal competitive interactions minimized the effect of centrality. Our model
analysis greatly illustrates the growing recognition in metacommunity ecology
that we must move toward more realistic landscapes \parencite{Gilarranz2012}.
For field ecologists, and particularly limnologists, our review emphasizes that
we need to go beyond geographic based analysis of $\beta$ diversity (e.g.
\parencite{Legendre2005}) to topological based analyses
\parencite{Peterson2013}.

The network approach to the study of spatially explicit landscapes is a major
advancement in metacommunity ecology. It is a first step to make the concept
operational because it accounts for more realistic landscape structures and
dispersal kernels. It makes a significant departure to simple island-mainland or
global dispersal approaches used previously (e.g. \parencite{Tilman1994a,
Hubbell2001, Mouquet2002}). But dispersal is also spatially explicit in a
lattice model and it does not make the landscape more realistic. We believe the
fundamental contribution of this approach is the consideration of spatial
heterogeneity of dispersal. In agreement with previous theoretical
\parencite{Economo2011, Desjardins2012b} and experimemental studies
\parencite{Carrara2012}, the simulations show that the degree centrality has a
significant impact on $\alpha$ diversity. Central nodes might also contribute
more to maintain $\gamma$ diversity, as they are essential for species to spread
throughout the landscape. The nodes could be potentially quantified as keystone
for the metacommunity \parencite{Mouquet2013}. Interestingly, but not
surprisingly, this effect is weaker with species sorting dynamics. We could even
hypothesize it will vanish in the extreme case of niche differentiation (with
low overlap for instance) and low mass effect. In this particular case, the
neutral versus niche comparison therefore illustrates that very strong unequal
competitive interactions could overwhelm the impact of dispersal.

The network approach and the comparison between metacommunity perspectives
reveal there could be spatial variation in coexistence mechanisms. If we take
the species-sorting perspective for instance, we find that $\alpha$ diversity is
higher in more central nodes. Since the environment is on average the same from
one patch to another, it implies that diversity in these communities is
maintained by a stronger mass effect. It results in spatial variation in the
relative importance of species-sorting, the mass effect and to a certain extent
the neutral drift. Because the degree centrality was the best variable
explaining diversity, we should expect the degree distribution to strongly
impact the relative contribution of these coexistence mechanisms. For a given
set of ecological processes and distribution of species traits, we might expect
the coexistence mechanisms to differ from one landscape to another.

We introduced this article arguing that neutral theory could be used as an
instrument to predict species distribution in spatially heterogeneous
landscapes. So far we have treated only theoretical models, but we could also
envision to parametrize them and simulate real landscapes. The recruitment
probabilities defined above could be used as statistical models (likelihood
functions) to fit to empirical data. Prior information could be used to define
apriori dispersal kernels and then fit the model as in \textcite{Gravel2008}.
The fit of metapopulation models to spatially explicit landscapes was pioneered
by \textcite{Hanski1998} and recently extended to species distribution models
including both species sorting and dispersal limitations
\parencite{Boulangeat2012}. Given the parametrization, one could run neutral
models to generate null hypotheses that could be eventually compared to observed
distribution. This would make a significant improvement over traditional null
models in ecology \parencite{Gotelli1996} in which there are no interactions and no dispersal
limitations.

The multivariate variance partitioning framework originally proposed by
\textcite{Borcard1992} and further developed by \textcite{Borcard2002} has been
widely used to quantify the relative importance of species sorting and dispersal
limitations in species distribution. This framework was originally proposed to
model species distribution as a function of environmental variables, taking into
account the spatial autocorrelation of species distribution
\parencite{Leduc1992, Borcard1992, Legendre1993}. This methodology has been
widely used over the last decade as a test of the neutral theory, its underlying
assumption and a quantification of dispersal limitations (e.g.
\textcite{Svenning2004, Hardy2004, Gilbert2004, Cottenie2005}. One problem of
this approach is however that it makes a weak test of neutrality
\parencite{McGill2003}, based on the description of spatial community structure,
rather than hypothesis testing. The different models we reviewed in this article
could be better employed if used to generate null expectations of species
distribution based on different hypotheses and then compare them. But most of
all, parametrized spatially explicit neutral models could be more useful if used
to predict biodiversity under different global change scenarios. For instance,
neutral models could be used to predict the consequences of habitat destruction,
fragmentation or a change in the connectivity matrix \parencite{Hubbell2008}.
The spatially explicit description of the landscape is a major improvement
toward that end, providing much flexibility in the scenarios that could be
explored.

Working with more complex representations on landscapes has several advantages.
Real landscapes are not flat geometric objects, they are highly structured and
diverse. This diverse structure has long been recognized as a key component of
diversity. In the 19th century, Moritz Wagner noted that patterns of rivers
\cite{coy04} could explain how beetles diversified. Yet, to this day, the
relationship between spatial structures and biodiversity is not resolved, but
networks provide powerful tools to analyze landscapes and generate testable
predictions on the relationship between community assembly and spatial
structures. For example, the neutral theory predict less diversity in isolated
communities and constant speciation (regardless of isolation) \parencite{Economo2008}.
Adding the effect of gene flow changes the predictions, but the neutral theory
do not predict adaptive radiation and might thus face problems for predicting
many long-term ecological processes.

Finally, our analyses emphasize the need to expand on the canonial neutral
theory. As pointed out by \textcite{Wootton2005}, most of the unexplained
deviation of empirical community from the prediction of accurately calibrated
neutral models can be attributed to non-competitive interactions. \textcite{Canard2012}
proposed that neutral processes can explain the network structure of trophic
interactions with a good accuracy. Incorporating reasonable complexity in the
mechanisms addressed by neutral models is not a theoreticians' exercise: it will
re-enforce the usefulness of the neutral theory as an operational concept,
specifically one that can be used to derive baseline predictions about (i) the
expected local species richness, and (ii) the expected species pool
dissimilarity at the between-site and regional scales. These predictions are the
benchmark against which empirical relev\'es of species richness and community
structure ought to be compared, and coming up with realistic parameters to
calibrate these models calls for a closer cooperation and dialogue between
theoreticians and empiricists.


%========================================================%
\section*{Acknowledgements}
DG received financial support from NSERC and Canada Research Chair program. TP
is supported by a MELS-FQRNT post-doctoral fellowship and PDP by a NSERC
fellowship.
\newpage

%========================================================%
\printbibliography

\newpage
%========================================================%
\section*{Figure legends}

%------------------------
\subsection*{Figure 1}
\textbf{Illustration of the four simulated landscapes}. The color code
represents the $\alpha$ diversity simulated with a neutral model, ranked from
the poorest (red) to the richest (blue). Parameters: $N = 25$, $r = 0.3 $, $S =
100$, $m = 0.2$, $M = 0.01$, $k = 0.1$, $J_x = 100$. Simulations run 1000 time
steps.

%------------------------
\subsection*{Figure 2}
\textbf{Relationship between $\alpha$ diversity and node centrality}. The upper
two panels are simulation results conducted with the random geometric graph
illustrated at Fig. 1 and the lower two panels are runs with the random tree
graph. Parameters:$S = 100$, $c = 0.4$, $e = 0.1$, $J_x = 100$, $m = 0.2$, $M =
0.01$, $k = 0.1$, $ u \epsilon [0,100]$, $b = 15$, $ E_x \epsilon [0,100]$,
$\overline{E_R} = 50$, $\sigma{E_R} = 5$. Simulations run 1000 time steps

%------------------------
\subsection*{Figure 3}
\textbf{Bray curtis dissimilarity as a function of Euclidean distance}. Parameters as in Fig. 2.

%------------------------
\subsection*{Figure 4}
\textbf{Bray curtis dissimilarity as a function of geodesic distance}. Parameters as in Fig. 2.

\newpage
%========================================================%
%------------------------
\subsection*{Figure 1}

\begin{figure}[ht!]
	\centering\includegraphics[width=0.3\textwidth]{Networks.pdf}
\end{figure}

\newpage

%------------------------
\subsection*{Figure 2}

\begin{figure}[ht!]
	\centering\includegraphics[width=0.75\textwidth]{Centrality.pdf}
\end{figure}

\newpage

%------------------------
\subsection*{Figure 3}

\begin{figure}[ht!]
	\centering\includegraphics[width=0.65\textwidth]{BetaGeoDist.png}
\end{figure}

\newpage

%------------------------
\subsection*{Figure 4}

\begin{figure}[ht!]
	\centering\includegraphics[width=0.65\textwidth]{BetaTopoDist.png}
\end{figure}

\newpage

%========================================================%

\begin{table*}[c]
	\centering
	\begin{tabular}{p{5cm}p{8cm}}
		\hline
		\textbf{Concept} & \textbf{Definition} \\
		\hline
		Path & A sequence of edges forming a sequence of nodes \\
		Connection & Two nodes are connected if there is a path between them \\
		Euclidean distance & Geographical distance between two nodes \\
		Geodesic distance & Length of the shortest path between two nodes \\

		\textbf{Network-level metrics} & \\
		Order & Total number of nodes \\
		Size & Total number of edges \\
		Connectivity & A measure of robustness: the minimum number of elements to remove to isolate the nodes \\
		Components &  The number of connected subsets\\

		\textbf{Node-level metrics} & \\
		Degree & The number of edges of a node \\
		Closeness centrality & Average geodesic distance between a node and all other vertex \\
		Eigenvector centrality & A measure of centrality based on the concept that connection to
		highly connected nodes are more important \\
		Betweenness centrality & The number of shortest paths from all nodes to all others that
		pass through that note \\
		\hline
	\end{tabular}
	\caption{Main descriptors of spatial networks used in this study.}
\end{table*}
\newpage

%------------------------

\begin{table*}[c]
	\centering
	\begin{tabular}{llccc}
	\hline
	& Definition & Patch dynamics & Neutral & Species-sorting \\
	\hline
	\textbf{Variables} &     &    &    &   \\
	$p$ & Occupancy  & X &    &   \\
	$N$ & Local population size &    & X & X \\
	$Z$ & Local rel. abund. &  & X & X \\
	$P$ & Rel. abund. in the neighborhood &  & X & X \\
	$s$ & Local species richness & X & X & X \\
	$d$ & Node degree & X & X & X \\
	$C$ & Prob. of a colonization event & X &  &  \\
	$I$ & Prob. of a colonization event & X &  &  \\
	$Pr$ & Recruitment prob. &  & X & X \\
	$\lambda$ & Survival prob. &  &  & X \\

	\textbf{Indices} & & & &\\
	$x,y$ & Node location & X & X & X \\
	$i,j$ & Species & X & X & X\\
	$n$ & Microsite & & & X \\

	\textbf{Parameters} & & & &\\
	$S$ & Size of regional species pool & X & X & X \\
	$c$ & Colonization prob. & X & & \\
	$e$ & Extinction prob. & X & & \\
	$J$ & Local carrying capacity & & X & X \\
	$m$ & immigration prob. from neigh. & & X & X \\
	$M$ & immigration prob. from metaco. & & X & X \\
	$k$ & Death prob. & & X & X \\
	$u$ & Niche optimum & & & X \\
	$b$ & Niche breadth & & & X \\
	$E$ & Microsite env. conditions & & & X \\
	$\overline{E}$ & Local env. average & & & X \\
	$\sigma$ & Local env. variance & & & X \\
	$\overline{E_R}$ & Regioal env. average & & & X \\
	$\sigma_R$ & Regional env. variance & & & X \\
	\hline
	\end{tabular}
	\caption{List of variables, indices and parameters from the three models}
\end{table*}
\newpage

\end{document}
%========================================================%
